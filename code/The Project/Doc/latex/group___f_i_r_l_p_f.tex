\hypertarget{group___f_i_r_l_p_f}{\section{F\-I\-R Lowpass Filter Example}
\label{group___f_i_r_l_p_f}\index{F\-I\-R Lowpass Filter Example@{F\-I\-R Lowpass Filter Example}}
}
\begin{DoxyParagraph}{Description\-: }

\end{DoxyParagraph}
\begin{DoxyParagraph}{}
Removes high frequency signal components from the input using an F\-I\-R lowpass filter. The example demonstrates how to configure an F\-I\-R filter and then pass data through it in a block-\/by-\/block fashion. 
\end{DoxyParagraph}
\begin{DoxyParagraph}{Algorithm\-:}

\end{DoxyParagraph}
\begin{DoxyParagraph}{}
The input signal is a sum of two sine waves\-: 1 k\-Hz and 15 k\-Hz. This is processed by an F\-I\-R lowpass filter with cutoff frequency 6 k\-Hz. The lowpass filter eliminates the 15 k\-Hz signal leaving only the 1 k\-Hz sine wave at the output. 
\end{DoxyParagraph}
\begin{DoxyParagraph}{}
The lowpass filter was designed using M\-A\-T\-L\-A\-B with a sample rate of 48 k\-Hz and a length of 29 points. The M\-A\-T\-L\-A\-B code to generate the filter coefficients is shown below\-: 
\begin{DoxyPre}
    h = fir1(28, 6/24);
\end{DoxyPre}
 The first argument is the \char`\"{}order\char`\"{} of the filter and is always one less than the desired length. The second argument is the normalized cutoff frequency. This is in the range 0 (D\-C) to 1.\-0 (Nyquist). A 6 k\-Hz cutoff with a Nyquist frequency of 24 k\-Hz lies at a normalized frequency of 6/24 = 0.\-25. The C\-M\-S\-I\-S F\-I\-R filter function requires the coefficients to be in time reversed order. 
\begin{DoxyPre}
    fliplr(h)
\end{DoxyPre}
 The resulting filter coefficients and are shown below. Note that the filter is symmetric (a property of linear phase F\-I\-R filters) and the point of symmetry is sample 14. Thus the filter will have a delay of 14 samples for all frequencies. 
\end{DoxyParagraph}
\begin{DoxyParagraph}{}
 
\end{DoxyParagraph}
\begin{DoxyParagraph}{}
The frequency response of the filter is shown next. The passband gain of the filter is 1.\-0 and it reaches 0.\-5 at the cutoff frequency 6 k\-Hz. 
\end{DoxyParagraph}
\begin{DoxyParagraph}{}
 
\end{DoxyParagraph}
\begin{DoxyParagraph}{}
The input signal is shown below. The left hand side shows the signal in the time domain while the right hand side is a frequency domain representation. The two sine wave components can be clearly seen. 
\end{DoxyParagraph}
\begin{DoxyParagraph}{}
 
\end{DoxyParagraph}
\begin{DoxyParagraph}{}
The output of the filter is shown below. The 15 k\-Hz component has been eliminated. 
\end{DoxyParagraph}
\begin{DoxyParagraph}{}

\end{DoxyParagraph}
\begin{DoxyParagraph}{Variables Description\-:}

\end{DoxyParagraph}
\begin{DoxyParagraph}{}
\begin{DoxyItemize}
\item {\ttfamily test\-Input\-\_\-f32\-\_\-1k\-Hz\-\_\-15k\-Hz} points to the input data \item {\ttfamily ref\-Output} points to the reference output data \item {\ttfamily test\-Output} points to the test output data \item {\ttfamily fir\-State\-F32} points to state buffer \item {\ttfamily fir\-Coeffs32} points to coefficient buffer \item {\ttfamily block\-Size} number of samples processed at a time \item {\ttfamily num\-Blocks} number of frames\end{DoxyItemize}

\end{DoxyParagraph}
\begin{DoxyParagraph}{C\-M\-S\-I\-S D\-S\-P Software Library Functions Used\-:}

\end{DoxyParagraph}
\begin{DoxyParagraph}{}

\begin{DoxyItemize}
\item \hyperlink{group___f_i_r_ga98d13def6427e29522829f945d0967db}{arm\-\_\-fir\-\_\-init\-\_\-f32()}
\item \hyperlink{group___f_i_r_gae8fb334ea67eb6ecbd31824ddc14cd6a}{arm\-\_\-fir\-\_\-f32()}
\end{DoxyItemize}
\end{DoxyParagraph}
{\bfseries  Refer } \hyperlink{arm_fir_example_f32_8c-example}{arm\-\_\-fir\-\_\-example\-\_\-f32.\-c} 